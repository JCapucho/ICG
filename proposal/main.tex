\documentclass{article}

\usepackage[a4paper, margin=2cm]{geometry}
\usepackage{graphicx}

\usepackage[T1]{fontenc}
\usepackage{fontspec}
\usepackage[sfdefault]{inter}

\title{Introduction to Computer Graphics: Project Proposal}
\author{João Capucho 113713}
\date{March 14, 2025}

\begin{document}

\maketitle

\section*{Introduction}

My project will be based around recreating the concept behind the \emph{Portal}
series of games. \emph{Portal} is a series of puzzle games where the player is
given access to a device capable of putting a pair of portals in surfaces that
instantly transport any object that comes into contact with one of them to the
other.

\begin{figure}[h]
	\centering
	\includegraphics[width=0.6\textwidth]{portal2.jpg}
	\caption{Image from the game \emph{Portal} 2}
\end{figure}

\section*{Expected results}

The final version is expected to implement a 3D environment where the user
controls a character that has collisions and follows some basic laws of
physics, like gravity and inertia. The user can then press a button to cause a
projectile to be spawned with an initial velocity and isn't affected by any
acceleration. Once it hits a surface this projectile will spawn one portal, and
when two are spawned they will link allowing the player to walk into one and
come out from the other.

The environment will be modeled with dynamic lights and using textures (taken
from the internet) using Physically Based Rendering (PBR) provided by three.js.
The character will use a model that contains animations (taken from the
internet) that will be played according to the user's input (idling, running,
falling, ...).

A UI for a main menu and pause state will also be developed.

\end{document}
