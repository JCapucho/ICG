\documentclass{article}

\usepackage[a4paper, margin=2cm]{geometry}
\usepackage{graphicx}
\usepackage{hyperref}

\usepackage[T1]{fontenc}
\usepackage{fontspec}
\usepackage[sfdefault]{inter}

\title{Introduction to Computer Graphics: Intermediate Report}
\author{João Capucho 113713}
\date{April 10, 2025}

\begin{document}

\maketitle

\section*{Introduction}

My project is based around recreating the concept behind the \emph{Portal}
series of games. \emph{Portal} is a series of puzzle games where the player is
given access to a device capable of putting a pair of portals in surfaces that
instantly transport any object that comes into contact with one of them to the
other.

\begin{figure}[h]
	\centering
	\includegraphics[width=0.6\textwidth]{portal2.jpg}
	\caption{Image from the game \emph{Portal} 2}
\end{figure}

The project doesn't yet have a name. The project is hosted at
\url{https://github.com/JCapucho/ICG} and a working version can be accessed
through \url{https://jcapucho.github.io/ICG/}.

\section*{Expected results}

The final version is expected to implement a 3D environment where the user
controls a character that has collisions and follows some basic laws of
physics, like gravity and inertia. The user can then press a button to cause a
projectile to be spawned with an initial velocity and isn't affected by any
acceleration. Once it hits a surface this projectile will spawn one portal, and
when two are spawned they will link allowing the player to walk into one and
come out from the other.

The environment will be modeled with dynamic lights and using textures (taken
from the internet) using Physically Based Rendering (PBR) provided by three.js.
The character will use a model that contains animations (taken from the
internet) that will be played according to the user's input (idling, running,
falling, ...).

A UI for a main menu and pause state will also be developed.

\section*{What is done}

Currently, the recursive portal rendering is mostly working. This required
setting up a camera per portal, that renders onto a texture instead of the
default framebuffer. This presented a cyclic dependency between the texture
that needs to be rendered as a portal and will be rendered to. To fix this an
intermediate texture is used that the camera renders to, and later is copied
into the texture that the portal draws.

The portal cameras also needed to react to the player's camera to simulate the
player "peeking" into the portal from different angles and positions. This was
implemented through some vector and quaternion math.

A physics engine \emph{Rapier} was also integrated for collision detection and
force generation. The player character is fully integrated into the physics
world, allowing it to move around and collide with floor and other objects.

\section*{What is missing}

The portal rendering still has some problems with frustum culling that causes
it to render objects behind the portal. This will need a new projection matrix
to be calculated per camera, to fix the clipping plane.

The portals need to be made interactive by allowing objects (such as the
player) to enter through one end and exit the other while conserving momentum.

The portals are statically created, while it's expected that the user can
create new portals by shooting projectiles.

Finally, a complete level showcasing the game mechanics needs to be
modeled and implemented.

\end{document}
